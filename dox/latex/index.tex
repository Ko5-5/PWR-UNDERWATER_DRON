\begin{DoxyAuthor}{Autor}
Michał Kos 
\end{DoxyAuthor}
\begin{DoxyDate}{Data}
22.\+05.\+2020 
\end{DoxyDate}
\begin{DoxyVersion}{Wersja}
0.\+1
\end{DoxyVersion}
Przedstawiony program jest pierwsza faza projektu modelowania drona, poruszajacego sie pod woda, przy pomocy jezyka C++ oraz programu Gnuplot. Program jest moja realizacja I fazy projektu, w ktorej dron podwodny porusza sie na pustej scenie. Wystepuje tylko tafla wody oraz dno, ktore ograniczaja jego ruch.\hypertarget{index_mozliwosci-dron}{}\section{Mozliwosci drona podwodnego}\label{index_mozliwosci-dron}
Obecna wersja programu pozwala na obrot drona wokol Osi Z jego ukladu lokalnego. Mozliwy jest rowniez ruch na wprost pod zadanym katem wznoszenia lub opadania.\hypertarget{index_ograniczenia-ruchu}{}\section{Ograniczenia ruchu}\label{index_ograniczenia-ruchu}
Program ma wbudowane ograniczenie ruchu drona w osi Z. \hyperlink{class_dron}{Dron} nie moze wzniesc sie wyzej niz wysokosc tafli wody, a w momencie zetkniecia z dnem, program informuje uzytkownika o kolizji i konczy swoje dzialanie. 